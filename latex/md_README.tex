\subsubsection*{Objetivos}

Los objetivos de esta práctica son\+:


\begin{DoxyItemize}
\item Que el alumnado conozca la herramienta Doxygen y utilice ese formato para la documentación de sus programas
\item Poner en práctica los conocimientos de programación estructurada
\item Poer en práctica conocimientos de programación Orientada a Objetos en C++
\item Practicar operaciones de entrada/salida (E/S) en ficheros de texto
\end{DoxyItemize}

\subsubsection*{Rúbrica de evaluacion de esta práctica}

Se señalan a continuación los aspectos más relevantes (la lista no es exhaustiva) que se tendrán en cuenta a la hora de evaluar esta práctica\+:


\begin{DoxyItemize}
\item El alumnado ha de acreditar conocimientos para trabajar con la shell de G\+N\+U/\+Linux en su VM
\item El código ha de estar escrito de acuerdo al estándar de la \href{https://google.github.io/styleguide/cppguide.html}{\tt Google C++ Style Guide}.
\item El alumnado ha de acreditar que es capaz de formatear su código usando V\+SC de acuerdo al estándar anterior
\item El alumnado ha de acreditar capacidad para editar ficheros remotos en la VM de la asignautra usando V\+SC
\item El programa desarrollado deberá compilarse utilizando la herramienta {\ttfamily make} y un fichero {\ttfamily Makefile}
\item El comportamiento del programa debe ajustarse a lo solicitado en este documento.
\item Ha de acreditarse capacidad para introducir cambios en el programa desarrollado.
\item Los comentarios del programa han de escribirse en formato Doxygen.
\item Modularidad\+: el programa ha de escribirse de modo que las diferentes funcionalidades que se precisen sean encapsuladas en funciones y/o métodos cuya extensión textual se mantenga acotada.
\item El programa ha de ser fiel al paradigma de programación orientada a objetos (O\+OP).
\end{DoxyItemize}

Si el alumnado tiene dudas respecto a cualquiera de estos aspectos, debiera acudir al foro de discusiones de la asignatura para plantearlas allı́. Se espera que, a través de ese foro, el alumnado intercambie experiencias y conocimientos, ayudándose mutuamente a resolver dichas dudas. También el profesorado de la asignatura intervendrá en las discusiones que pudieran suscitarse, si fuera necesario.

\subsubsection*{Introducción}

Las palabras de Fibonacci, F\textsubscript{1}, F\textsubscript{2}, F\textsubscript{3}, ... se definen como\+:
\begin{DoxyItemize}
\item F\textsubscript{1} = \char`\"{}a\char`\"{}
\item F\textsubscript{2} = \char`\"{}b\char`\"{}
\item Para cualquier n ≥ 3, F\textsubscript{n} es el resultado de concatenar F\textsubscript{n-\/2} con F\textsubscript{n-\/1}.
\end{DoxyItemize}

Las primeras siete palabras de la secuencia de Fibonacci son\+: F\textsubscript{1} = \char`\"{}a\char`\"{}, F\textsubscript{2} = \char`\"{}b\char`\"{}, F\textsubscript{3} = \char`\"{}ab\char`\"{}, F\textsubscript{4} = \char`\"{}bab\char`\"{}, F\textsubscript{5} = \char`\"{}abbab\char`\"{}, F\textsubscript{6} = \char`\"{}bababbab\char`\"{}, F\textsubscript{7} = \char`\"{}abbabbababbab\char`\"{}.

\subsubsection*{Ejercicio práctico}

Diseñar e implementar en C++ un programa {\ttfamily fibonacci\+\_\+words.\+cc} que dada una secuencia de palabras indique si son palabras de Fibonacci o no. Para aquellas que lo sean, el programa ha de indicar su posición en la secuencia.

Realice un enfoque orientado a objetos para el desarrollo de su programa\+: identifique objetos en el problema que se propone resolver y modele esos objetos mediante las clases y métodos que considere más adecuados.

El programa tomará su entrada de un fichero de texto, y su salida la escribirá igualmente en otro. Si el fichero de entrada contiene las siguientes palabras\+:


\begin{DoxyCode}
a
b
ba
aaa
bb
bababbab
\end{DoxyCode}


y el fichero se llama {\ttfamily input.\+txt}, cuando el programa se invoque como

{\ttfamily \$$>$ ./fibonacci\+\_\+words input.\+txt output.\+txt}

el contenido del fichero {\ttfamily output.\+txt} debiera ser el que se muestra a continuación\+:


\begin{DoxyCode}
a is the word number 1
b is the word number 2
ba is not a Fibonacci word
aaa is not a Fibonacci word
bb is not a Fibonacci word
bababbab is the word number 6
\end{DoxyCode}


Cuando el programa se invoque sin pasarle parámetros en la línea de comandos, deberá finalizar su ejecución, imprimiendo en pantalla un mensaje de una única línea que indique la forma correcta de ejecución. Si el programa se invoca pasándole por línea de comandos el parámetro {\ttfamily -\/-\/help}\+:

{\ttfamily \$$>$ ./fibonacci\+\_\+words -\/-\/help}

El programa deberá igualmente finalizar su ejecución pero indicando en este caso el modo correcto de invocarlo así como una breve explicación de la finalidad del programa.

\subsubsection*{Doxygen}

\href{https://www.doxygen.nl/index.html}{\tt Doxygen} es un generador de documentación para código fuente que permite generar la documentación completa de un proyecto software. La documentación está escrita en en el código, y por lo tanto es relativamente fácil de mantener actualizada. Doxygen puede hacer referencias cruzadas entre la documentación y el código, de modo que posibilita que el lector de un documento pueda referirse fácilmente al código fuente de la aplicación. Doxygen es un software libre y al igual que Javadoc extrae la documentación de los comentarios de los ficheros fuente. La herramienta puede generar documentación en H\+T\+ML así como en otros formatos.

A partir de esta práctica, todos los programas que se desarrollen deberán contener su documentación escrita en formato Doxygen. Se propone que no se ocupen tanto del uso de la herramienta en sí, sino de que la documentación de su código se escriba de aucuerdo al estándar de esa herramienta. Para iniciarse en el uso de Doxygen se propone que estudie las siguientes referencias\+:

\subsubsection*{Referencias}


\begin{DoxyItemize}
\item \href{https://www.doxygen.nl/index.html}{\tt Doxygen}
\item \href{https://www.doxygen.nl/manual/docblocks.html#specialblock}{\tt Documenting the code}
\item \href{https://www-numi.fnal.gov/offline_software/srt_public_context/WebDocs/doxygen-howto.html}{\tt How to document your code for doxygen}
\item \href{https://devblogs.microsoft.com/cppblog/visual-studio-code-c-extension-july-2020-update-doxygen-comments-and-logpoints/}{\tt Doxygen comments in V\+SC} 
\end{DoxyItemize}